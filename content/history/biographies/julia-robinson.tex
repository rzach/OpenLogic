% Part: history 
% Chapter: biographies 
% Section: julia-robinson
\documentclass[../../../include/open-logic-section]{subfiles}

\begin{document}

\olfileid{his}{bio}{rob}

\olsection{Julia Robinson}

\olphoto{robinson-julia}{Julia Robinson}

Julia Bowman Robinson was an American mathematician. She is known
mainly for her work on decision problems, and most famously for her
contributions to the solution of Hilbert's tenth problem. Robinson was
born in St.~Louis, Missouri, on December 8, 1919. Robinson recalls
being intrigued by numbers already as a child \citep[4]{Reid1986}.
At age nine she contracted scarlet fever and suffered from several
recurrent bouts of rheumatic fever. This forced her to spend much of
her time in bed, putting her behind in her education.  Although she
was able to catch up with the help of private tutors, the physical
effects of her illness had a lasting impact on her life.

Despite her childhood struggles, Robinson graduated high school with
several awards in mathematics and the sciences. She started her
university career at San Diego State College, and transferred to the
University of California, Berkeley, as a senior. There she was
influenced by the mathematician Raphael Robinson. They became good
friends, and married in 1941. As a spouse of a faculty member,
Robinson was barred from teaching in the mathematics department at
Berkeley. Although she continued to audit mathematics classes, she
hoped to leave university and start a family. Not long after her
wedding, however, Robinson contracted pneumonia. She was told that
there was substantial scar tissue build up on her heart due to the
rheumatic fever she suffered as a child. Due to the severity of the
scar tissue, the doctor predicted that she would not live past forty
and she was advised not to have children \citep[13]{Reid1986}.

Robinson was depressed for a long time, but eventually decided to
continue studying mathematics. She returned to Berkeley and completed
her PhD in 1948 under the supervision of Alfred Tarski. The
first-order theory of the real numbers had been shown to be decidable
by Tarski, and from G\"odel's work it followed that the first-order
theory of the natural numbers is undecidable.  It was a major open
problem whether the first-order theory of the rationals is decidable
or not. In her thesis \citeyearpar{Robinson1949}, Robinson proved that
it was not. 

Interested in decision problems, Robinson next attempted to find a
solution to Hilbert's tenth problem. This problem was one of a famous
list of 23 mathematical problems posed by David Hilbert in 1900. The
tenth problem asks whether there is an algorithm that will answer, in
a finite amount of time, whether or not a polynomial equation with
integer coefficients, such as $3x^2 - 2y +3 = 0$, has a solution in
the integers. Such questions are known as \emph{Diophantine problems}.
After some initial successes, Robinson joined forces with Martin Davis
and Hilary Putnam, who were also working on the problem. They
succeeded in showing that exponential Diophantine problems (where the
unknowns may also appear as exponents) are undecidable, and showed
that a certain conjecture (later called ``J.R.'')  implies that
Hilbert's tenth problem is undecidable
\citep{DavisPutnamRobinson1961}.  Robinson continued to work on the
problem throughout the 1960s.  In 1970, the young Russian
mathematician Yuri Matijasevich finally proved the J.R. hypothesis.
The combined result is now called the
Matijasevich--Robinson--Davis--Putnam theorem, or MRDP theorem for short.
Matijasevich and Robinson became friends and collaborated on several
papers. In a letter to Matijasevich, Robinson once wrote that
``actually I am very pleased that working together (thousands of miles
apart) we are obviously making more progress than either one of us
could alone'' \citep[45]{Matijasevich1992}.

Robinson was the first female president of the American Mathematical
Society, and the first woman to be elected to the National
Academy of Science. She died on July 30, 1985 at the age of 65 after
being diagnosed with leukemia.

\begin{reading}
Robinson's mathematical papers are available in her \textit{Collected
  Works} \citep{Robinson1996}, which also includes a reprint of her
National Academy of Sciences biographical memoir
\citep{Feferman1994}. Robinson's older sister Constance Reid published
an ``Autobiography of Julia,'' based on interviews \citep{Reid1986},
as well as a full memoir \citep{Reid1996}. A short documentary about
Robinson and Hilbert's tenth problem was directed by George Csicsery
\citep{Csicsery2016}. For a brief memoir about Yuri Matijasevich's
collaborations with Robinson, and her influence on his work, see
\citep{Matijasevich1992}.
\end{reading}

\end{document}
