% Part: sets-functions-relations
% Chapter: relations
% Section: orders

\documentclass[../../../include/open-logic-section]{subfiles}

\begin{document}

\olfileid{sfr}{rel}{ord}
\olsection{Orders}

\begin{explain}
Many of our comparisons involve describing some objects as being
``less than'', ``equal to'', or ``greater than'' other objects, in a
certain respect. These involve \emph{order} relations. But there are
different kinds of order relations. For instance, some require that
any two objects be comparable, others don't. Some include identity
(like~$\le$) and some exclude it (like~$<$). It will help us to have a
taxonomy here.
\end{explain}

\begin{defn}[Preorder]
A relation which is both reflexive and transitive is called a
\emph{preorder.}  
\end{defn}

\begin{defn}[Partial order]
A preorder which is also anti-symmetric is called a
\emph{partial order}.
\end{defn}

\begin{defn}[Linear order]\ollabel{def:linearorder}
A partial order which is also connected is called a
\emph{total order} or \emph{linear order.}
\end{defn}

\begin{ex}
Every linear order is also a partial order, and every partial order is
also a preorder, but the converses don't hold. The universal relation
on~$A$ is a preorder, since it is reflexive and transitive. But, if
$A$ has more than one !!{element}, the universal relation is not
anti-symmetric, and so not a partial order.
\end{ex}

\begin{ex}
Consider the \emph{no longer than} relation $\preccurlyeq$
on~$\Bin^*$: $x \preccurlyeq y$ iff $\len{x} \le \len{y}$. This is a
preorder (reflexive and transitive), and even connected, but not a
partial order, since it is not anti-symmetric. For instance, $01
\preccurlyeq 10$ and $10 \preccurlyeq 01$, but $01 \neq 10$.
\end{ex}

\begin{ex}
An important partial order is the relation $\subseteq$ on a set of
sets. This is not in general a linear order, since if $a \neq b$ and
we consider $\Pow{\{a, b\}} = \{\emptyset, \{a\}, \{b\}, \{a,b\}\}$,
we see that $\{a\} \nsubseteq \{b\}$ and $\{a\} \neq \{b\}$ and $\{b\}
\nsubseteq \{a\}$.
\end{ex}

\begin{ex}
The relation of \emph{divisibility without remainder} gives us a
partial order which isn't a linear order. For integers $n$, $m$, we
write $n\mid m$ to mean $n$ (evenly) divides $m$, i.e., iff there is
some integer~$k$ so that $m=kn$. On $\Nat$, this is a partial order, but not a
linear order: for instance, $2\nmid3$ and also $3\nmid2$. Considered
as a relation on $\Int$, divisibility is only a preorder since
it is not anti-symmetric: $1\mid-1$ and $-1\mid1$ but $1\neq-1$.
\end{ex}

\begin{defn}[Strict order]
A \emph{strict order} is a relation which is irreflexive, asymmetric,
and transitive.
\end{defn}

\begin{defn}[Strict linear order]\ollabel{def:strictlinearorder}
A strict order which is also connected is called a 
\emph{strict total order} or \emph{strict linear order.}
\end{defn}

\begin{ex}
$\le$ is the linear order corresponding to the strict linear
order~$<$. $\subseteq$ is the partial order corresponding to the
strict order~$\subsetneq$.
\end{ex}

Any strict order $R$ on~$A$ can be turned into a partial order by
adding the diagonal $\Id{A}$, i.e., adding all the pairs~$\tuple{x,
x}$.  (This is called the \emph{reflexive closure} of~$R$.)
Conversely, starting from a partial order, one can get a strict order
by removing~$\Id{A}$. These next two results make this precise.

\begin{prop}\ollabel{prop:stricttopartial}
If $R$ is a strict order on~$A$, then $R^+ = R \cup \Id{A}$ is a
partial order. Moreover, if $R$ is a strict linear order, then $R^+$ is 
a linear order.
\end{prop}

\begin{proof}
Suppose $R$ is a strict order, i.e., $R \subseteq A^2$ and $R$ is
irreflexive, asymmetric, and transitive. Let $R^+ = R \cup \Id{A}$. We
have to show that $R^+$ is reflexive, anti-symmetric, and transitive.

$R^+$ is clearly reflexive, since $\tuple{x, x} \in \Id{A} \subseteq
R^+$ for all $x \in A$. 

To show $R^+$ is anti-symmetric, suppose for reductio that $R^+xy$ and
$R^+yx$ but $x \neq y$. Since $\tuple{x,y} \in R \cup \Id{A}$, but
$\tuple{x, y} \notin \Id{A}$, we must have $\tuple{x, y} \in R$, i.e.,
$Rxy$. Similarly,~$Ryx$. But this contradicts the assumption
that $R$ is asymmetric.

To establish transitivity, suppose that $R^+xy$ and $R^+yz$. If both
$\tuple{x, y} \in R$ and $\tuple{y,z} \in R$, then $\tuple{x, z} \in
R$ since $R$~is transitive. Otherwise, either $\tuple{x, y} \in
\Id{A}$, i.e., $x = y$, or $\tuple{y, z} \in \Id{A}$, i.e., $y = z$.
In the first case, we have that $R^+yz$ by assumption, $x = y$, hence
$R^+xz$. Similarly in the second case. In either case, $R^+xz$, thus,
$R^+$ is also transitive.

Concerning the ``moreover'' clause, suppose that $R$ is also connected. 
So for all $x \neq y$, either $Rxy$ or~$Ryx$, i.e., either 
$\tuple{x, y} \in R$ or $\tuple{y, x} \in R$. Since $R \subseteq R^+$, 
this remains true of $R^+$, so $R^+$ is connected as well.
\end{proof}

\begin{prop}\ollabel{prop:partialtostrict}
If $R$ is a partial order on~$A$, then $R^- = R \setminus \Id{A}$ is a
strict order. Moreover, if $R$ is a linear order, then $R^-$ is a strict 
linear order.
\end{prop}

\begin{proof}
This is left as an exercise.
\end{proof}

\begin{prob}
Give a proof of \olref[sfr][rel][ord]{prop:partialtostrict}. 
\end{prob}

The following simple result establishes that strict linear orders
satisfy an extensionality-like property:

\begin{prop}\ollabel{prop:extensionality-strictlinearorders}
If $<$ is a strict linear order on $A$, then:
\[
  (\forall a, b \in A)((\forall x \in A)(x < a \liff x < b) \lif a = b).
\]
\end{prop}

\begin{proof}
Suppose $(\forall x \in A)(x < a \liff x < b)$. If $a < b$, then $a <
a$, contradicting the fact that $<$ is irreflexive; so $a \nless b$.
Exactly similarly, $b \nless a$. So $a = b$, as $<$ is connected.
\end{proof}

\end{document}
