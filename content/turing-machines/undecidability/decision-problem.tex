% Part: turing-machines
% Chapter: undecidability
% Section: unsolvability-decision-problem

\documentclass[../../../include/open-logic-section]{subfiles}

\begin{document}

\olfileid{tur}{und}{dec}
\olsection{The Decision Problem}

We say that first-order logic is \emph{decidable} iff there is an
effective method for determining whether or not a given !!{sentence}
is valid. As it turns out, there is no such method: the problem of
deciding validity of first-order sentences is unsolvable.

In order to establish this important negative result, we prove that
the decision problem cannot be solved by a Turing machine.  That is,
we show that there is no Turing machine which, whenever it is started
on a tape that contains a first-order !!{sentence}, eventually halts
and outputs either $1$ or~$0$ depending on whether the
!!{sentence} is valid or not. By the Church--Turing thesis, every
function which is computable is Turing computable. So if this
``validity function'' were effectively computable at all, it would be
Turing computable. If it isn't Turing computable, then, it also cannot
be effectively computable.

Our strategy for proving that the decision problem is unsolvable is to
reduce the halting problem to it.  This means the following: We have
proved that the function~$h(e,w)$ that halts with output~$1$ if the
Turing machine described by~$e$ halts on input~$w$ and outputs~$0$
otherwise, is not Turing computable.  We will show that if there were
a Turing machine that decides validity of first-order sentences, then
there is also Turing machine that computes~$h$.  Since $h$ cannot be
computed by a Turing machine, there cannot be a Turing machine that
decides validity either.

The first step in this strategy is to show that for every input~$w$
and a Turing machine~$M$, we can effectively describe !!a{sentence}
$!T(M, w)$ representing the instruction set of~$M$ and the input~$w$
and !!a{sentence}~$!E(M, w)$ expressing ``$M$ eventually halts'' such
that:
\begin{quote}
  $\Entails !T(M, w) \lif !E(M,w)$ iff $M$ halts for input~$w$.
\end{quote}
The bulk of our proof will consist in describing these sentences
$!T(M, w)$ and~$!E(M, w)$ and in verifying that $!T(M, w) \lif !E(M, w)$
is valid iff $M$~halts on input~$w$.

\end{document}
