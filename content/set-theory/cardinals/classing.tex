\documentclass[../../../include/open-logic-section]{subfiles}

\begin{document}

\olfileid{sth}{cardinals}{classing}
\olsection{Finite, \usetoken{S}{enumerable}, \usetoken{S}{nonenumerable}}

Now that we have been introduced to cardinals, it is worth spending a
little time talking about different varieties of cardinals;
specifically, finite, !!{enumerable}, and !!{nonenumerable} cardinals.

Our first two results entail that the finite cardinals will be exactly
the finite ordinals, which we defined as our \emph{natural numbers}
back in \olref[z][infinity-again]{defnomega}: 

\begin{prop}\ollabel{finitecardisoequal}
Let $n, m \in \omega$. Then $n = m$ iff $\cardeq{n}{m}$.
\end{prop}

\begin{proof}
\emph{Left-to-right} is trivial. To prove \emph{right-to-left},
suppose $\cardeq{n}{m}$ although $n \neq m$. By Trichotomy, either $n
\in m$ or $m \in n$; suppose $n \in m$ without loss of generality.
Then $n \subsetneq m$ and there is !!a{bijection} $f \colon m \to n$,
so that $m$ is Dedekind infinite, contradicting
\olref[z][infinity-again]{naturalnumbersarentinfinite}.
\end{proof}

\begin{cor}\ollabel{naturalsarecardinals}
If $n \in \omega$, then $n$ is a cardinal. 
\end{cor}

\begin{proof}
Immediate.
\end{proof}
\noindent
It also follows that several reasonable notions of what it might mean
to describe a cardinal as ``finite'' or ``infinite'' coincide:
\begin{thm}\ollabel{generalinfinitycharacter}For any set $A$, the following are equivalent:
\begin{enumerate}
	\item\ollabel{card:notinomega} $\card{A} \notin \omega$, i.e.,
	$A$ is not a natural number;
	\item\ollabel{card:omegaplus} $\omega \leq \card{A}$;
	\item\ollabel{card:infinite} $A$ is Dedekind infinite.
\end{enumerate}
\end{thm}

\begin{proof}
From \olref[ord-arithmetic][using-addition]{ordinfinitycharacter},
\olref[cardinals][cardsasords]{lem:CardinalsBehaveRight}, and
\olref{naturalsarecardinals}. 
\end{proof}

This licenses the following \emph{definition} of some notions which we
used rather informally in \olref[sfr][][]{part}:

\begin{defn}\ollabel{defnfinite}
We say that $A$ is \emph{finite} iff $\card{A}$ is a natural number,
i.e., $\card{A} \in \omega$. Otherwise, we say that $A$ is
\emph{infinite}.
\end{defn}
\noindent 
But note that this definition is presented against the background of
$\ZFC$. After all, we needed Well-Ordering to guarantee that every set
has a cardinality. And indeed, without Well-Ordering, there can be a
set which is neither finite nor Dedekind infinite. We will return to
this sort of issue in \olref[choice][]{chap}. For now, we continue to
rely upon Well-Ordering.

Let us now turn from the finite cardinals to the infinite cardinals.
Here are two elementary points:

\begin{cor}\ollabel{omegaisacardinal}
$\omega$ is the least infinite cardinal. 
\end{cor}

\begin{proof}
$\omega$ is a cardinal, since $\omega$ is Dedekind infinite and if
$\cardeq{\omega}{n}$ for any $n \in \omega$ then $n$ would be Dedekind
infinite, contradicting
\olref[z][infinity-again]{naturalnumbersarentinfinite}. Now
$\omega$ is the least infinite cardinal by definition. 
\end{proof}

\begin{cor}
Every infinite cardinal is a limit ordinal.
\end{cor}

\begin{proof}
Let $\alpha$ be an infinite successor ordinal, so $\alpha = \beta
\ordplus 1$ for some $\beta$. By \olref{finitecardisoequal}, $\beta$
is also infinite, so $\cardeq{\beta}{\beta \ordplus 1}$ by
\olref[ord-arithmetic][using-addition]{ordinfinitycharacter}. Now
$\card{\beta} = \card{\beta\ordplus 1} = \card{\alpha}$ by
\olref[cardinals][cardsasords]{lem:CardinalsBehaveRight}, so
that $\alpha \neq \card{\alpha}$.
\end{proof}

Now, as early as \olref[sfr][siz][enm-alt]{defn:enumerable}, we flagged we
can distinguish between !!{enumerable} and !!{nonenumerable} infinite
sets. That definition naturally leads to the following:

\begin{prop}
$A$ is !!{enumerable} iff $\card{A} \leq \omega$, and $A$ is
!!{nonenumerable} iff $\omega < \card{A}$.
\end{prop}

\begin{proof}
By Trichotomy, the two claims are equivalent, so it suffices to prove
that $A$ is !!{enumerable} iff $\card{A} \leq \omega$. For
\emph{right-to-left}: if $\card{A} \leq \omega$, then
$\cardle{A}{\omega}$ by
\olref[cardinals][cardsasords]{lem:CardinalsBehaveRight} and
\olref{omegaisacardinal}. For \emph{left-to-right}: suppose $A$ is
!!{enumerable}; then by \olref[sfr][siz][enm-alt]{defn:enumerable} there
are three possible cases:
\begin{enumerate}
	\item if $A = \emptyset$, then $\card{A} = 0 \in \omega$, by
	\olref{naturalsarecardinals} and
	\olref[cardinals][cardsasords]{lem:CardinalsBehaveRight}.
	\item if $\cardeq{n}{A}$, then $\card{A} = n \in \omega$, by
	\olref{naturalsarecardinals} and
	\olref[cardinals][cardsasords]{lem:CardinalsBehaveRight}.
	\item if $\cardeq{\omega}{A}$, then $\card{A} = \omega$, by \olref{omegaisacardinal}.
\end{enumerate}
So in all cases, $\card{A} \leq \omega$. 
\end{proof}
\noindent
Indeed, $\omega$ has a special place. Whilst there are many countable ordinals:

\begin{cor}
$\omega$ is the only !!{enumerable} infinite cardinal.
\end{cor}

\begin{proof}
Let $\cardfont{a}$ be !!a{enumerable} infinite cardinal. Since
$\cardfont{a}$ is infinite, $\omega \leq \cardfont{a}$. Since
$\cardfont{a}$ is !!a{enumerable} cardinal, $\cardfont{a} =
\card{\cardfont{a}} \leq \omega$. So $\cardfont{a} = \omega$ by
Trichotomy. \end{proof}

Of course, there are infinitely many cardinals. So we might ask:
\emph{How many cardinals are there?} The following results show that
we might want to reconsider that question.

\begin{prop}\ollabel{unioncardinalscardinal}
If every member of $X$ is a cardinal, then $\bigcup X$ is a cardinal.
\end{prop}

\begin{proof}
It is easy to check that $\bigcup X$ is an ordinal. Let $\alpha \in
\bigcup X$ be an ordinal; then $\alpha \in \cardfont{b} \in X$ for
some cardinal $\cardfont{b}$. Since $\cardfont{b}$ is a cardinal,
$\cardless{\alpha}{\cardfont{b}}$. Since $\cardfont{b} \subseteq
\bigcup X$, we have $\cardle{\cardfont{b}}{\bigcup X}$, and so
$\cardneq{\alpha}{\bigcup X}$. Generalising, $\bigcup X$ is a
cardinal.
\end{proof} 

\begin{thm}\ollabel{lem:NoLargestCardinal}
There is no largest cardinal.
\end{thm}

\begin{proof}
For any cardinal $\cardfont{a}$, Cantor's Theorem
(\olref[sfr][siz][car]{thm:cantor}) and
\olref[cardinals][cardsasords]{lem:CardinalsExist} entail that
$\cardfont{a} < \card{\Pow{\cardfont{a}}}$.
\end{proof}

\begin{thm}
The set of all cardinals does not exist.
\end{thm}

\begin{proof}
For reductio, suppose $C = \Setabs{\cardfont{a}}{\cardfont{a} \text{
is a cardinal}}$. Now $\bigcup C$ is a cardinal by
\olref{unioncardinalscardinal}, so by \olref{lem:NoLargestCardinal}
there is a cardinal $\cardfont{b} > \bigcup C$. By definition
$\cardfont{b} \in C$, so $\cardfont{b} \subseteq \bigcup{C}$, so that
$\cardfont{b} \leq \bigcup C$, a contradiction.
\end{proof}

You should compare this with both Russell's Paradox and Burali-Forti. 

\end{document}