% part: intuitionistic-logic
% chapter: introduction
% section: natural-deduction

\documentclass[../../../include/open-logic-section]{subfiles}

\begin{document}

\olfileid{int}{int}{ntd}

\olsection{Natural Deduction}

Natural deduction without the $\FalseCl$ rules is a standard
!!{derivation} system for intuitionistic logic. We repeat the rules
here and indicate the motivation using the BHK interpretation.  In
each case, we can think of a rule which allows us to conclude that if
the premises have constructions, so does the conclusion.

Since natural deduction !!{derivation}s have undischarged assumptions,
we should consider such !!a{derivation}, say, of $!A$ from
!!{undischarged} assumptions~$\Gamma$, as a function that turns
constructions of all $!B \in \Gamma$ into a construction of~$!A$.  If
there is !!a{derivation} of~$!A$ from no !!{undischarged} assumptions,
then there is a construction of~$!A$ in the sense of the BHK
interpretation. For the purpose of the discussion, however, we'll
suppress the~$\Gamma$ when not needed.

An assumption~$!A$ by itself is !!a{derivation} of~$!A$ from the
!!{undischarged} assumption~$!A$. This agrees with the
BHK-interpretation: the identity function on constructions turns any
construction of~$!A$ into a construction of~$!A$.

\subsection{Conjunction}

\begin{defish}
\AxiomC{$!A$}
\AxiomC{$!B$}
\RightLabel{\Intro{\land}}
\BinaryInfC{$!A \land !B$}
\DisplayProof
\hfill
\begin{tabular}{l}
\AxiomC{$!A \land !B$}
\RightLabel{\Elim{\land}}
\UnaryInfC{$!A$}
\DisplayProof
\\[3ex]
\AxiomC{$!A \land !B$}
\RightLabel{\Elim{\land}}
\UnaryInfC{$!B$}
\DisplayProof
\end{tabular}
\end{defish}

Suppose we have constructions $N_1$, $N_2$ of $!A_1$ and $!A_2$,
respectively.  Then we also have a construction $!A_1 \land !A_2$, namely the
pair $\tuple{N_1, N_2}$.

A construction of $!A_1 \land !A_1$ on the BHK interpretation is a pair
$\tuple{N_1, N_2}$. So assume we have such a pair. Then we also have a
construction of each conjunct: $N_1$ is a construction of~$!A_1$ and $N_2$ is a
construction of $!A_2$.

\subsection{Conditional}

\begin{defish}
  \AxiomC{$\Discharge{!A}{u}$}
  \DeduceC{$!B$}
  \DischargeRule{$\Intro{\lif}$}{u}
  \UnaryInfC{$!A \lif !B$}
  \DisplayProof
\hfill
  \AxiomC{$!A \lif !B$}
  \AxiomC{$!A$}
  \RightLabel{$\Elim{\lif}$}
  \BinaryInfC{$!B$}
  \DisplayProof
\end{defish}

If we have !!a{derivation} of~$!B$ from !!{undischarged}
assumption~$!A$, then there is a function~$f$ that turns constructions
of~$!A$ into constructions of~$!B$. That same function is a
construction of $!A \lif !B$. So, if the premise of $\Intro\lif$ has a
construction conditional on a construction of~$!A$, the conclusion $!A
\lif !B$ has a construction.

On the other hand, suppose there are constructions~$N$ of $!A$ and $f$
of $!A \lif !B$. A construction of $!A \lif !B$ is a function that
turns constructions of $!A$ into constructions of~$!B$. So, $f(N)$ is
a construction of~$!B$, i.e., the conclusion of $\Elim\lif$ has a
construction.

\subsection{Disjunction}

\begin{defish}
\begin{tabular}{l}
\AxiomC{$!A$}
\RightLabel{\Intro{\lor}}
\UnaryInfC{$!A \lor !B$}
\DisplayProof
\\[3ex]
\AxiomC{$!B$}
\RightLabel{\Intro{\lor}}
\UnaryInfC{$!A \lor !B$}
\DisplayProof
\end{tabular}
\hfill
\AxiomC{$!A \lor !B$}
\AxiomC{$\Discharge{!A}{n}$}
\DeduceC{$!C$}
\AxiomC{$\Discharge{!B}{n}$}
\DeduceC{$!C$}
\DischargeRule{\Elim{\lor}}{n}
\TrinaryInfC{$!C$}
\DisplayProof
\end{defish}

If we have a construction~$N_i$ of~$!A_i$ we can turn it into a
construction~$\tuple{i, N_i}$ of~$!A_1 \lor !A_2$. On the other hand, suppose
we have a construction of~$!A_1 \lor !A_2$, i.e., a pair $\tuple{i, N_i}$
where $N_i$ is a construction of~$!A_i$, and also functions $f_1$, $f_2$,
which turn constructions of $!A_1$, $!A_2$, respectively, into constructions
of~$!C$. Then $f_i(N_i)$ is a construction of~$!C$, the conclusion
of~$\Elim\lor$.

\subsection{Absurdity}

\begin{defish}
  \AxiomC{$\lfalse$}
  \RightLabel{\FalseInt}
  \UnaryInfC{$!A$}
  \DisplayProof
\end{defish}

If we have !!a{derivation} of~$\lfalse$ from !!{undischarged}
assumptions~$!B_1$, \dots, $!B_n$, then there is a function~$f(M_1,
\dots, M_n)$ that turns constructions of $!B_1$, \dots, $!B_n$ into a
construction of~$\lfalse$. Since $\lfalse$ has no construction, there
cannot be any constructions of all of $!B_1$, \dots, $!B_n$
either. Hence, $f$ also has the property that \emph{if} $M_1$, \dots,
$M_n$ are constructions of $!B_1$, \dots, $!B_n$, respectively,
\emph{then} $f(M_1, \dots, M_n)$ is a construction of~$!A$.

\subsection{Rules for $\lnot$}

Since $\lnot !A$ is defined as $!A \lif \lfalse$, we strictly speaking
do not need rules for~$\lnot$. But if we did, this is what they'd look
like:

\begin{defish}
\AxiomC{$\Discharge{!A}{n}$}
\noLine
\DeduceC{$\lfalse$}
\DischargeRule{\Intro{\lnot}}{n}
\UnaryInfC{$\lnot !A$}
\DisplayProof
\hfill
\AxiomC{$\lnot !A$}
\AxiomC{$!A$}
\RightLabel{\Elim{\lnot}}
\BinaryInfC{$\lfalse$}
\DisplayProof
\end{defish}


\subsection{Examples of \usetoken{P}{derivation}}

\begin{enumerate}
\item $\Proves !A \lif (\lnot !A \lif \lfalse)$, i.e., $\Proves !A
  \lif ((!A \lif \lfalse) \lif \lfalse)$
  \begin{prooftree}
    \AxiomC{$\Discharge{!A}{2}$}
    \AxiomC{$\Discharge{!A \lif \lfalse}{1}$}
    \RightLabel{\Elim\lif}
    \BinaryInfC{$\lfalse$}
    \DischargeRule{\Intro\lif}{1}
    \UnaryInfC{$(!A \lif \lfalse)\lif \lfalse$}
    \DischargeRule{\Intro\lif}{2}
    \UnaryInfC{$!A \lif (!A \lif \lfalse) \lif \lfalse$}
  \end{prooftree}

\item $\Proves ((!A \land !B) \lif !C) \lif (!A \lif (!B \lif !C))$
  \begin{prooftree}
    \AxiomC{$\Discharge{(!A \land !B) \lif !C}{3}$}
    \AxiomC{$\Discharge{!A}{2}$}
    \AxiomC{$\Discharge{!B}{1}$}
    \RightLabel{\Intro\land}
    \BinaryInfC{$!A \land !B$}
    \RightLabel{\Elim\lif}
    \BinaryInfC{$!C$}
    \DischargeRule{\Intro\lif}{1}
    \UnaryInfC{$!B \lif !C$}
    \DischargeRule{\Intro\lif}{2}
    \UnaryInfC{$!A \lif (!B \lif !C)$}
    \DischargeRule{\Intro\lif}{3}
    \UnaryInfC{$((!A \land !B) \lif !C) \lif (!A \lif (!B \lif !C))$}
  \end{prooftree}

\item $\Proves \lnot(!A \land \lnot !A)$, i.e., $\Proves (!A \land (!A
  \lif \lfalse))\lif \lfalse$
  \begin{prooftree}
    \AxiomC{$\Discharge{!A \land (!A \lif \lfalse)}{1}$}
    \RightLabel{\Elim\land}
    \UnaryInfC{$!A \lif \lfalse$}
    \AxiomC{$\Discharge{!A \land (!A \lif \lfalse)}{1}$}
    \RightLabel{\Elim\land}
    \UnaryInfC{$!A$}
    \RightLabel{\Elim\lif}
    \BinaryInfC{$\lfalse$}
    \DischargeRule{\Intro\lif}{1}
    \UnaryInfC{$(!A \land (!A \lif \lfalse))\lif \lfalse$}
  \end{prooftree}

\item $\Proves \lnot\lnot(!A \lor \lnot !A)$, i.e., $\Proves ((!A \lor
  (!A \lif \lfalse))\lif \lfalse)\lif \lfalse$
  \begin{prooftree}\hskip-3em
    \AxiomC{$\Discharge{(!A \lor (!A \lif \lfalse))\lif \lfalse}{2}$}
    \AxiomC{$\Discharge{(!A \lor (!A \lif \lfalse))\lif \lfalse}{2}$}
    \AxiomC{$\Discharge{!A}{1}$}
    \RightLabel{\Intro\lor}
    \UnaryInfC{$!A \lor (!A \lif \lfalse)$}
    \RightLabel{\Elim\lif}
    \BinaryInfC{$\lfalse$}
    \DischargeRule{\Intro\lif}{1}
    \UnaryInfC{$!A \lif \lfalse$}
    \RightLabel{\Intro\lor}
    \UnaryInfC{$!A \lor (!A \lif \lfalse)$}
    \RightLabel{\Elim\lif}
    \insertBetweenHyps{\hskip -4em}
    \BinaryInfC{$\lfalse$}
    \DischargeRule{\Intro\lif}{2}
    \UnaryInfC{$((!A \lor (!A \lif \lfalse))\lif \lfalse)\lif \lfalse$}
  \end{prooftree}
\end{enumerate}

\begin{prop}
  If $\Gamma \Proves !A$ in intuitionistic logic, $\Gamma \Proves !A$ in
  classical logic. In particular, if $!A$ is an intuitionistic
  theorem, it is also a classical theorem.
\end{prop}

\begin{proof}
  Every natural deduction rule is also a rule in classical natural
  deduction, so every !!{derivation} in intuitionistic logic is also
  !!a{derivation} in classical logic.
\end{proof}

\begin{prob}
  Give !!{derivation}s in intuitionistic logic of the following !!{formula}s:
  \begin{enumerate}
    \item $(\lnot !A \lor !B) \lif (!A \lif !B)$
    \item $\lnot\lnot\lnot !A \lif \lnot !A$
    \item $\lnot\lnot (!A \land !B) \liff (\lnot\lnot !A\land \lnot \lnot !B)$
    \item $\lnot (!A \lor !B) \liff (\lnot !A \land \lnot !B)$
    \item $(\lnot !A \lor \lnot !B) \lif \lnot (!A \land !B)$
    \item $\lnot\lnot ( !A \land !B) \lif  (\lnot\lnot !A \lor
    \lnot\lnot !B)$
  \end{enumerate}
\end{prob}

\end{document}
