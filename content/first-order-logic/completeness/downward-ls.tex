% Part: first-order-logic
% Chapter: completeness
% Section: downward-ls

\documentclass[../../../include/open-logic-section]{subfiles}

\begin{document}

\olfileid{fol}{com}{dls}
\olsection{The L\"owenheim--Skolem Theorem}

The L\"owenheim--Skolem Theorem says that if a theory has an infinite
model, then it also has a model that is at most !!{denumerable}. An
immediate consequence of this fact is that first-order logic cannot
express that the size of !!a{structure} is !!{nonenumerable}: any
!!{sentence} or set of !!{sentence}s satisfied in all
!!{nonenumerable} !!{structure}s is also satisfied in some
!!{enumerable} structure.

\begin{thm} 
\ollabel{thm:downward-ls} If $\Gamma$ is consistent then it has
!!a{enumerable} model, i.e., it is satisfiable in !!a{structure}
whose domain is either finite or !!{denumerable}.
\end{thm}

\begin{proof}
If $\Gamma$ is consistent, the !!{structure}~$\Struct M$ delivered by
the proof of the completeness theorem has a domain $\Domain{M}$ that
is no larger than the set of the terms of the language~$\Lang L$. So
$\Struct M$ is at most !!{denumerable}.
\end{proof}

\begin{thm}
\ollabel{noidentity-ls} If $\Gamma$ is a consistent set of !!{sentence}s
in the language of first-order logic without identity, then it has
!!a{denumerable} model, i.e., it is satisfiable in !!a{structure}
whose domain is infinite and !!{enumerable}.
\end{thm}

\begin{proof}
If $\Gamma$ is consistent and contains no sentences in which identity
appears, then the !!{structure}~$\Struct M$ delivered by the proof of
the completeness theorem has a domain $\Domain{M}$ identical to the set
of terms of the language~$\Lang L'$. So $\Struct{M}$ is
!!{denumerable}, since $\Trm[L']$ is.
\end{proof}

\begin{ex}[Skolem's Paradox]
Zermelo--Fraenkel set theory~$\Log{ZFC}$ is a very powerful framework
in which practically all mathematical statements can be expressed,
including facts about the sizes of sets. So for instance, $\Log{ZFC}$
can prove that the set~$\Real$ of real numbers is !!{nonenumerable},
it can prove Cantor's Theorem that the power set of any set is larger
than the set itself, etc.  If $\Log{ZFC}$ is consistent, its models
are all infinite, and moreover, they all contain !!{element}s about
which the theory says that they are !!{nonenumerable}, such as the
element that makes true the theorem of~$\Log{ZFC}$ that the power set
of the natural numbers exists. By the L\"owenheim--Skolem Theorem,
$\Log{ZFC}$ also has !!{enumerable} models---models that contain
``!!{nonenumerable}'' sets but which themselves are !!{enumerable}.
\end{ex}

\end{document}
